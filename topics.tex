Viability of emotionally adaptive user interface in the context of increasing performance of e-learning.

- test if 2 different interfaces (oriented for positive valence (colourful,joyful) and negative valence(strict, b/w) ) and 2 different moods (pos and neg) lead to different performance on a task. Condition the user to a mood before presenting with the task.
- task should be oriented for a specific mood. 
- Performance can be measured in time spent, successful rate, mistakes per session etc.
- find a standardized test to use (proven best for a certain kind of valence/arousal state)
- discuss viability of real-time or in-between-session measurement of mood to adjust interface




Todo: 
- hypothesen formulieren und studien design formulieren.





Merits and possibilities of emotional self-regulation in the context of an e-learning plattform

base on the research that people tend to emotionally regulate themselves when made aware of the emotion and which emotion is best.
present the user with historical data about performance and emotional state
let the user choose alert points when their mood reaches threshhold, allow them to self regulate to optimize their own performance