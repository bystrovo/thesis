

\pagebreak

\section{Background and motivation}

E-learning providers are looking to improve the learning experience of their users and make progress as effective as possible. 
During a usual learning session each learner is subject to a range of volatile emotional states that help or hinder their learning success. 
A range of factors and stimuli, both internal and external, can influence emotions. 
When we take this knowledge about emotions into account a new way to manage a learning session opens. We can incorporate this knowledge into learning sessions and provide a more appropriate task or interface for the learner.

The goal of the paper is to examine a relation between emotional aspect of a user interface in an e-learning system and its effect on performance of the learner.


\begin{center}
	\includegraphics[width=200px]{images/relation1.png}
\end{center}
 
There is strong evidence of the surrounding environment having an influence on emotion \cite{Johnson2000, Arockiam2013, Bertamini2013}. This includes, for example, an e-learning system on the screen in front of the learner. In a similar fashion several studies have shown correlation between emotion and cognition.

\begin{center}
\includegraphics[width=300px]{images/relation2.png}
\end{center}

There is a logical argument of the existence of a transitive relation between these parameters, which could confirm the dependency of the edge variables. 
I.e. exposure to several interfaces each with a different emotional charge during an on-line lesson should lead to a difference in performance when working on the same task.
Insufficient research confirming this connection and explaining the effects has been published yet. 

In this paper I would like to explore to which extent the final parameter "learning success" can be influenced with the limited surface of contact that can be addressed through a learning interface on the screen.

Ethical, legal and social considerations are also part of my thesis.

\section{Approach and methods}

I propose a study that challenges human short-term memory, attention and creativity in a set of tasks. I will create a hypothetical learning environment that presents one set of activities in 2 different interfaces, intending to evoke different emotions from the user. In a blind study each subject will be assigned one of the interfaces and complete provided tasks (further described below).

Two fundamental questions to be answered:
\begin{itemize}
	\item Have these 2 interfaces achieved the expected emotional response?
	\item Did the results of the task vary between interfaces, as expected in accordance to the emotional response
	% sind die Ergebnisse der aufgaben unterschiedlich ausgefallen, wie erwartet in Abhängigkeit von der emotionalen Reaktion.
\end{itemize}

\subsection{Study design}

\begin{figure}
	\centering
	\includegraphics[width=0.25\linewidth]{images/valence-arousal.png}
	\caption[Emotions in 2 dimensions]{Emotions in 2 dimensions}
	\label{fig:valence-arousal}
\end{figure}

A usual way to represent an emotional state is through 2 attributes: arousal and valence. Arousal can range from 0 (low) to 1 (high) and valence from -1 (negative) to 1 (positive) (See figure \ref{fig:valence-arousal}). 

\textbf{1. Preconditioning:} At the beginning of the experiment each test user is preconditioned with the help of visual materials to be in one of 4 states:
 \textbf{1}: Positive valence / high arousal; 
 \textbf{2}: Negative valence / high arousal;
 \textbf{3}: Positive valence / low arousal;
 \textbf{4}: Negative valence / low arousal;

Studies \citationneeded propose effective sets of emotional materials to achieve this.

\begin{figure}
	\begin{center}
		\includegraphics[width=1\textwidth]{images/study_design3.png}
		\caption{Study design\label{fig:scaled_diss}}
	\end{center}
\end{figure}

\textbf{2. Validation:} A short emotional self-awareness questionnaire will be used to validate, whether pre-conditioning has had sufficient and expected effect on the subject.

\textbf{3. Tasks:} A prepared task sequence (\ref{task_sequence}) will focus on testing short-term memory and creative/analytical problems. 
50\% of each group are presented with a task-set through Interface0 and 50\% would complete the tasks through Interface1.

\begin{itemize}
	\item \textbf{Inteface 0 (I0):} Default interface with low emotional capacity
	\item \textbf{Inteface 1 (I1):} Optimized interface with elements tailored for high arousal and positive valence (potentially optimal for both memory and analytical tasks) \citationneeded
\end{itemize}

Both interface versions would provide equal usability features. 
The difference is in the emotional charge which stems from \underline{language used}, \underline{forms and shapes}, \underline{colors} \citationneeded and potentially an emotional image.

\textbf{4. Revalidation:} After completing all tasks the subject will repeatedly fill out the questionnaire from step 2 to capture any changes to their emotion, caused by the interface and task.

Effectively the study will divide subjects into 8 groups. 
It is my goal to have each testing sequence completed within about 20 minutes time to keep the subjects' time investment low.

\subsection{Measuring performance} \label{measuring}

During each task several parameters will be recorded for each subject to create a comprehensive task performance metric. 
Examples include: rate of clicks per minute, mouse movements, rate of correct answers and time elapsed until completion. 
Mouse movement and video recordings could allow further qualitative analysis and provide room for more insight.

\subsection{Task Sequence} \label{task_sequence}

Psychological and cognitive studies have developed a variety of tests evaluating abilities of human mind. 
They differ in the art of activity and their focus, for example some concentrate on memory, creative thinking, analytical thinking, solving mathematical problems, imagination or orientation capabilities.

For this study I found several potentially fitting tests:

- \textbf{a classic memory game}. The goal is to find all pairs of images on a raster of (ex. 10 by 10) tiles. 
Only 2 pieces can be turned over at once. 
After which they turn back around and another try of finding a pair starts. 
In the context of this study several adjustments will reduce the ratio of luck/chance involved in choosing correct tiles: Showing all cards at the beginning, discarding the correct tile choices at the end of the game when few tiles left unturned, and others.

- \textbf{Remote Associates Test}. Originally published in 1962 by \cite[p.226 ff]{Mednick1962}, this test evaluates individual creativity. 
The test subject is presented with a set of 3 words and they must find a fourth word that is in some way connected to all three words.

\subsection{Evaluation}

Usually a quantitative analysis can yield better results with a larger number of subjects.
I expect to attain a minimum of 15 subjects per group (15 * 8 = 120 ). But the more I can get - the better.
To simplify acquisition of test subjects I will create a testing framework in the form of a web-application (explained section \ref{implementation}) and release it through online media to be completed remotely.
Additional steps within the application will be taken to minimize risk of noise and false results that come from this type of testing. Part of the study will be conducted in a controlled environment on university grounds.

\section{Implementation} \label{implementation}

To facilitate the study I will build a web application (based on reactjs), runnable on desktop and tablet devices. Data emitted from the application will be saved in a remote database (mongodb) for the purposes of evaluation.

Some open source software might be used and integrated into the system when applicable. E-learning tasks development is not an integral part of this thesis, although potential technological bases for integration of emotion information in e-learning systems will be discussed.

No information should be stored locally on the device after finishing a session.

\section{Implications of the study}

If this study can demonstrate a significant difference in performance across groups, we can conclude that emotional features of digital user interface can have a noticeable impact on the way people consume information, memorize and analyze problems.
It will pave the way for further studies being done exactly which aspects of an interface have the biggest impact and how to design for sustained attention and performance.

E-learning providers can use these findings to better structure their online lessons and invest in design that supports the learner.

Further these findings can be adapted and generalized to make use of within other applications in software, digital media and similar outlets.

