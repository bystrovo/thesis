\section {Problem description}

E-Learning systems provide a convenient way to convey knowledge. They allow a student to learn material at their own preferred pace. 
Current state of-the-art e-learning systems furthermore include a training component with intelligent tailored guidance for completion. 

Such systems come close to an actual learning environment in a school, but are limited due to their technology and remote nature. More precisely, these systems lack an “emotional” component that can be expressed in a more personal setting. Devices like eye-movement trackers, pulse-, and various other vital-sign-trackers can be introduced to improve on this limitation. Such additional information can create an environment where a better and more intelligent personal one-on-one guidance system becomes a possibility.

Learning performance analysis is an ever-present subject of educational research. Understanding how and when we learn and memorize material in the most efficient way would allow to tune the software and increase their effectiveness. Studies have shown that emotion and learning have a strong correlation between one another, yet not much research has been done in the field of creating interfaces to arouse and support an emotional state that is beneficial for learning. 

There are several considerations in developing an unobtrusive and supportive extension to an e-learning platform. These include:
\begin{itemize}
	\item Technological complexity of providing near-real-time data to the system and being able to compute context-sensitive and time-relevant actions based on received input.
	\item Psychological approach that considers human traits and reactions to stimuli, making correct assumptions of the human state of mind
	\item Visual language that responds to the input in an unobtrusive and supporting way, guiding but not disturbing the learning process
\end{itemize}

Bisherige Forschung.
Forschungslücke behaubten
Lücke füllen

\section{Goals of the paper}

The first goal of the paper is to conceptually create a prototype that would propose certain actions based on possible input data in the system. Furthermore, to create a technological infrastructure for consuming data from the available sources and transforming it in a way appropriate to the target e-learning system. Third, analyse and propose different approaches to guiding a lesson with additionally available data and to build a working prototype module for the target e-learning system including visualisation of historical lesson performance and real-time lesson guidance system.

\section{Approach and methods}

The first main section of this paper will focus on visual presentation of historical and real-time data to provide users of the e-learning system with best appropriate information during a lesson and after it for the purposes of analysis.

The second section will focus on technology needed to integrate and run on the system of the end-user (a web-browser)

\section{Implementation}

\section{Scheduling}
February – Research and defining the structure as well as concrete goal
March, April – concept development and research basis, writing
End of April registration of Thesis.
May, June – writing and describing concept (concept at this point is done and developed)
July, Aug – possible revisions and adjustments based on feedback and further findings (user testing)

\section{Sources}

Human Computer Interaction, Alan Dix, 2003, Prentice Hall
ISBN-13: 978-0130461094

Information Dashboard Design: Displaying Data for At-A-Glance Monitoring, Stephen Few, 2013, Analytics Press
ISBN-13: 978-1938377006

