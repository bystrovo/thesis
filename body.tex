\section {Problem description}

E-Learning systems provide a convenient, scalable way to convey and share knowledge. They allow a student to learn material at their own preferred pace. 
Current state-of-the-art e-learning systems furthermore include a training component with guidance that helps to complete a task-set posed to the user.

These systems come close to an actual learning environment in a school, but are limited due to their technological limitations and remote nature. More precisely, these systems lack an “emotional” component that can be expressed in a more personal setting with a tutor. Devices like eye-movement trackers, pulse-, and various other vital-sign-trackers can be introduced to improve on this limitation. Such additional information can create an environment where a better and more intelligent, more personal one-on-one guidance system becomes a possibility.

Learning performance analysis is an ever-present subject of educational research. Understanding how and when we learn and memorize material in the most efficient way can allow to tune learning systems and increase their effectiveness. Studies have shown that emotion is central in the cognitive process \cite{ORegan2003}, and feeling can influence the learning advancement \cite{Hawkins2017}.

Past research highlights the importance of the state of learners mind for a successful learning process, yet there has been insufficient research on user interfaces, that are suitable for e-learning systems, and their capacity to arouse and support an emotional and cognitive state, that is beneficial for learning.

%Bisherige Forschung.
%Forschungslücke behaubten
%Lücke füllen

\section{Goals of the paper}

\begin{itemize}
	\item Theoretical fundamentals
	\item Analyse available data and formats
	\item Create a Prototype (Front-End and Middleware to Backend) (implementation part)
	\subitem[-] A framework for inclusion of the module in any e-learning and common data format
	\item Develop a testing framework to measure the prototypes effectiveness and value for the user
\end{itemize}

The paper will explore the processes of consuming knowledge in an e-learning environment, methods for extracting and understanding the emotional and cognitive state of the learner (may be not this part, could be too far in the backend), as well as provide suggestions about guiding and influencing the state to achieve better learning performance with UX and UI techniques.

The first goal of the paper is to conceptually create a prototype that would intelligently propose certain actions, assuming available input data in the system.  
Furthermore, to create a technological infrastructure for consuming data from the available sources and transforming it in a way appropriate to the target e-learning system.

Third, to analyse and propose approaches to guiding a lesson with additionally available data and to build a working prototype module for the target e-learning system including visualisation of historical lesson performance and real-time lesson guidance system.

Ethical, legal and social issues.

\section{Approach and methods}

The first section of the paper will focus on visual presentation of historical and real-time data to provide users of the e-learning system with best appropriate information during a lesson and after it for the purposes of analysis.

I will explore the possibilities of supported and automatic decision making by the system to best decide which content to deliver to the user. Furthermore, I will propose ways of extracting emotional state of the user with direct user input to augment sensor data provided by the tracker.

The second section will focus on technology needed to integrate and run on the system of the end-user (a web-browser) and develop a suitable api to support several requirements of the analytical system.

\section{Implementation}

There are several considerations in developing an unobtrusive and supportive extension to an e-learning platform. These include:

\begin{itemize}
	\item Technological complexity of providing near-real-time data to the system and being able to compute context-sensitive and time-relevant actions based on received input.
	\item Psychological approach that considers human traits and reactions to stimuli, making correct assumptions of the human state of mind
	\item Visual language that responds to the input in an unobtrusive and supporting way, guiding but not disturbing the learning  \cite{dix2003human,few2013information}
\end{itemize}



\section{Scheduling}
\begin{itemize}
	\item February, March – Research and defining the structure as well as concrete goal
	\item March, April – concept development and research basis, writing
	\item End of May registration of Thesis.
	\item May, June – writing and describing concept (concept at the end of the period is done and developed)
	\item July, Aug – writing analysis, possible revisions and adjustments based on feedback and further findings (possibly user testing)
\end{itemize}

